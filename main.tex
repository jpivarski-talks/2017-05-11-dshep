\documentclass{beamer}

\mode<presentation>
{
  \usetheme{default}
  \usecolortheme{default}
  \usefonttheme{default}
  \setbeamertemplate{navigation symbols}{}
  \setbeamertemplate{caption}[numbered]
  \setbeamertemplate{footline}[page number]
  \setbeamercolor{frametitle}{fg=white}
  \setbeamercolor{footline}{fg=black}
} 

\usepackage[english]{babel}
\usepackage[utf8x]{inputenc}
\usepackage{tikz}
\usepackage{listings}
\usepackage{courier}
\usepackage{array}
\usepackage{bold-extra}
\usepackage{minted}
\usepackage{transparent}

\xdefinecolor{darkblue}{rgb}{0.1,0.1,0.7}
\xdefinecolor{darkgreen}{rgb}{0,0.5,0}
\xdefinecolor{darkgrey}{rgb}{0.35,0.35,0.35}
\xdefinecolor{darkorange}{rgb}{0.8,0.5,0}
\xdefinecolor{darkred}{rgb}{0.7,0,0}
\xdefinecolor{dianablue}{rgb}{0.18,0.24,0.31}
\definecolor{commentgreen}{rgb}{0,0.6,0}
\definecolor{stringmauve}{rgb}{0.58,0,0.82}

\lstset{ %
  backgroundcolor=\color{white},      % choose the background color
  basicstyle=\ttfamily\small,         % size of fonts used for the code
  breaklines=true,                    % automatic line breaking only at whitespace
  captionpos=b,                       % sets the caption-position to bottom
  commentstyle=\color{commentgreen},  % comment style
  escapeinside={\%*}{*)},             % if you want to add LaTeX within your code
  keywordstyle=\color{blue},          % keyword style
  stringstyle=\color{stringmauve},    % string literal style
  showstringspaces=false,
  showlines=true
}

\lstdefinelanguage{scala}{
  morekeywords={abstract,case,catch,class,def,%
    do,else,extends,false,final,finally,%
    for,if,implicit,import,match,mixin,%
    new,null,object,override,package,%
    private,protected,requires,return,sealed,%
    super,this,throw,trait,true,try,%
    type,val,var,while,with,yield},
  otherkeywords={=>,<-,<\%,<:,>:,\#,@},
  sensitive=true,
  morecomment=[l]{//},
  morecomment=[n]{/*}{*/},
  morestring=[b]",
  morestring=[b]',
  morestring=[b]"""
}

\title[2017-05-11-dshep]{\bf \LARGE Data Plumbing}
\author{Jim Pivarski}
\institute{Princeton University -- DIANA}
\date{May 11, 2017}

\begin{document}

\logo{\pgfputat{\pgfxy(0.11, 8)}{\pgfbox[right,base]{\tikz{\filldraw[fill=dianablue, draw=none] (0 cm, 0 cm) rectangle (50 cm, 1 cm);}}}\pgfputat{\pgfxy(0.11, -0.6)}{\pgfbox[right,base]{\tikz{\filldraw[fill=dianablue, draw=none] (0 cm, 0 cm) rectangle (50 cm, 1 cm);}\includegraphics[height=0.99 cm]{diana-hep-logo.png}\tikz{\filldraw[fill=dianablue, draw=none] (0 cm, 0 cm) rectangle (4.9 cm, 1 cm);}}}}

\usebackgroundtemplate{{\transparent{0.15}\includegraphics[width=\paperwidth,height=\paperheight]{plumbing.jpg}}}

\begin{frame}
  \titlepage
\end{frame}

\usebackgroundtemplate{}

\logo{\pgfputat{\pgfxy(0.11, 8)}{\pgfbox[right,base]{\tikz{\filldraw[fill=dianablue, draw=none] (0 cm, 0 cm) rectangle (50 cm, 1 cm);}\includegraphics[height=1 cm]{diana-hep-logo.png}}}}

% Uncomment these lines for an automatically generated outline.
%\begin{frame}{Outline}
%  \tableofcontents
%\end{frame}

%%%%%%%%%%%%%%%%%%%%%%%%%%%%%%%%%%%%%%%%%%%%%%%%%%%%%%%

\begin{frame}{Why?}
We're all here because we think that HEP can benefit from machine learning techniques.

\vfill
\uncover<2->{The most advanced techniques are being developed outside of HEP using (what are becoming) industry standard tools.}

\vfill
\uncover<3->{\textcolor{darkblue}{Problem:} not all of them interoperate with our HEP protocols and formats.}

\vfill
\uncover<4->{\textcolor{darkblue}{This talk is about moving data among formats and frameworks.}}
\end{frame}

\begin{frame}{}

\only<1>{\mbox{\hspace{-1 cm}\includegraphics[width=1.2\linewidth]{diana-hep.png}}}
\only<2>{\mbox{\hspace{-1 cm}\includegraphics[width=1.2\linewidth]{diana-hep2.png}}}

\end{frame}

\begin{frame}{Goals of this talk}
\Large
\begin{itemize}\setlength{\itemsep}{0.5 cm}
\item \textcolor{darkblue}{make \underline{you} aware of what's possible}
\item \textcolor{darkblue}{introduce some software tools}
\item \textcolor{darkblue}{invite you to tell \underline{me} what you need.}
\end{itemize}
\end{frame}

\begin{frame}{}
\begin{center}
\LARGE \textcolor{darkblue}{root\_numpy}
\end{center}
\end{frame}

\begin{frame}{}
\begin{center}
\LARGE \textcolor{darkblue}{direct to Numpy}
\end{center}
\end{frame}

\begin{frame}{}
\begin{center}
\LARGE \textcolor{darkblue}{industry standard formats}

\vspace{0.25 cm}
\textcolor{darkblue}{\Large Avro/ProtoBuf/Thrift, Parquet, Arrow/Feather}
\end{center}
\end{frame}

\begin{frame}{}
\begin{center}
\LARGE \textcolor{darkblue}{Spark and the JVM}
\end{center}
\end{frame}

\begin{frame}{}
\begin{center}
\LARGE \textcolor{darkblue}{in-memory sharing}
\end{center}
\end{frame}

\begin{frame}{}
\begin{center}
\LARGE \textcolor{darkblue}{moving fitted models}
\end{center}
\end{frame}

\begin{frame}{Conclusing remarks}

\end{frame}

\end{document}
